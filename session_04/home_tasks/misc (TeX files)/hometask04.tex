\documentclass[12pt]{article}

\usepackage[unicode]{hyperref}

\usepackage{amsmath, amsthm, amssymb, amsfonts}
\usepackage{mathtext, mathtools}

\usepackage[T1, T2A]{fontenc}
\usepackage[utf8]{inputenc}
\usepackage[english, russian]{babel}

\usepackage{indentfirst}
\usepackage{paracol}

\usepackage{geometry}
\geometry{a4paper,
	total={170mm,257mm},left=2cm,right=2cm,
	top=2cm,bottom=2cm}

\usepackage{graphicx}

\usepackage{titleps}
\newpagestyle{main}{
	\setheadrule{0.4pt}
	\sethead{лево}{центр}{право}
	\setfootrule{0.4pt}
	\setfoot{left}{\thepage}{право}
}

\DeclareMathOperator{\sign}{sign}
\DeclareMathOperator{\sigmoid}{sigmoid}
\DeclareMathOperator*{\argmax}{argmax}

\newenvironment{eq_array}{\begin{equation*}\begin{array}{l}}{\end{array}\end{equation*}}

\begin{document}
	
	\tableofcontents
	
	\section{SCHIZZOIDEA}
		Напишем калькулятор произведения матриц.
		
		Тема вообще не для 10-11 класса, но она, как мне кажется, чисто идейно несложная.
		А для практики работы со списками и индексами - вообще самое то, имхо.
	
	\section{Матричная алгербра}
	\subsection{Определение матрицы}
		Матрица $A = (a_{ij})$ в математике - таблица размером $n$ строк и $m$ столбцов.
		Матрица $3\times2$ - таблица в 3 строки, в каждой - по 2 колонки.
		
	\subsection{Определение произведения матриц}
		Произведением матрицы $A=(a_{ij})$ размера $n*m$ и матрицы $B=(b_{ij})$ размера $m*k$ является матрица $C=(c_{ij})$, где $c_{ij}$ - скалярное произведение $i$-ой строки A на $j$-ый столбец B. Виз лежит в конце PDF-ки, не волнуйтесь.
	
		Какие есть условия у произведения?
		\begin{enumerate}
			\item Члены операции - матрицы
			\item Количество столбцов матрицы слева от оператора равно количеству строк матрицы справа от оператора. Из определения это видно
		\end{enumerate}

	\section{<<ТЗ>>}
		Что сейчас могу требовать от вас в такой schizzoprogram:
		\begin{enumerate}
			\item Проверку на равенство столбцов количеству строк
			\item Вывод результата произведения матриц
		\end{enumerate}
	
	\subsection{Post Scriptum}
	<<Требование>> - если будет желание поботать Data Science, вам есть смысл сейчас:
	\begin{itemize}
		\item Разобраться в индексах элементов матриц
		\item Хранить матрицы как массив столбцов.
		
		Это абсолютно неинтуитивно. И NumPy (модуль для задротов-нердов), и обычные математики сначала как бы записывают обозначают количество строк, а потом строки <<делят>> на колонки. 
		
		Но т.к. Data Science на практике - это работа с Pandas, и, в частности, \href{https://pandas.pydata.org/docs/reference/api/pandas.DataFrame.html#pandas.DataFrame}{pandas.DataFrame}, которые:
		\begin{quotation}\begin{center}
			\textit{сan be thought of as a dict-like container for Series objects.}
		\end{center}\end{quotation}
		более правильно хранить матрицы как массив столбцов.
	\end{itemize}
	
	\section{Приложение. Выкладка произведений матриц}
		\[	A
			\begin{pmatrix*}[c]
				1 & 2 & 3 \\
				0 & 4 & 5
			\end{pmatrix*}
			\times
			B
			\begin{pmatrix*}[c]
				-1 & 1 & 4 & 2 \\
				0 & 1 & -1 & 3 \\
				2 & 3 & -2 & 1
			\end{pmatrix*}
			= C
			\begin{pmatrix*}[c]
				5 & 12 & -4 & 11 \\
				10 & 19 & -14 & 17
			\end{pmatrix*}
		\]
		Например, элемент $c_{22}$ был получен следующим образом:
		\begin{eq_array}
			0  \cdot 1  + 4 \cdot 1 + 5 \cdot 3 = 19 \\
			a_{21} \cdot b_{21} + a_{22} \cdot b_{22} + a_{23} \cdot b_{23} = 19
		\end{eq_array}
		\[\]
\end{document}